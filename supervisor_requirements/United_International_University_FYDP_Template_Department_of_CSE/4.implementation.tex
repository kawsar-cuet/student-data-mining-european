\chapter{Implementation and Experimental Setup}

This chapter presents technical implementation details including software stack specifications, hardware configuration, training procedures with hyperparameter tuning results, computational requirements, and reproducibility provisions ensuring replicability of this research.

\section{Software Stack and Development Environment}

\subsection{Programming Language and Libraries}

\begin{table}[h]
\centering
\caption{Software and Library Specifications}
\label{tab:software}
\begin{tabular}{llc}
\toprule
\textbf{Component} & \textbf{Software/Library} & \textbf{Version} \\
\midrule
Programming Language & Python & 3.10+ \\
Deep Learning & PyTorch & 2.8.0 \\
ML Algorithms & Scikit-learn & 1.4.0 \\
 & XGBoost & 2.0.3 \\
Data Processing & Pandas & 2.2.0 \\
 & NumPy & 1.26.0 \\
Visualization & Matplotlib & 3.8.0 \\
 & Seaborn & 0.13.0 \\
LLM API & OpenAI API & 1.12.0 \\
Interpretability & SHAP & 0.44.0 \\
\bottomrule
\end{tabular}
\end{table}

\subsection{Hardware Configuration}

\begin{table}[h]
\centering
\caption{Hardware Specifications and Requirements}
\label{tab:hardware}
\begin{tabular}{ll}
\toprule
\textbf{Component} & \textbf{Specification} \\
\midrule
Processor & Intel Core i7-12700K or equivalent \\
RAM & 32GB DDR4 \\
Storage & 500GB SSD \\
GPU & Optional (CPU-only for reproducibility) \\
\bottomrule
\end{tabular}
\end{table}

\section{Model Training Procedure}

\subsection{Training Pipeline and Hyperparameter Tuning}

Systematic grid search across multiple hyperparameter dimensions:

\textbf{Learning Rates Tested:} [0.0001, 0.001, 0.01]
\textbf{Batch Sizes Tested:} [16, 32, 64]
\textbf{Dropout Rates:} [0.1, 0.2, 0.3]
\textbf{Hidden Layer Dimensions:} Multiple architectures

\textbf{Total Configurations Evaluated:} 1,728 (PPN: 432, DPN-A: 648, HMTL: 648)
\textbf{Total Training Time:} 48.3 hours (PPN: 14.2h, DPN-A: 18.6h, HMTL: 15.5h)

\subsection{Optimal Hyperparameter Configurations}

\textbf{PPN Optimal Configuration:}
\begin{itemize}
\item Learning Rate: 0.001
\item Batch Size: 32
\item Dropout: 0.3 $\to$ 0.2 $\to$ 0.1 (progressive)
\item Validation F1-Macro: 0.745
\end{itemize}

\textbf{DPN-A Optimal Configuration:}
\begin{itemize}
\item Learning Rate: 0.001
\item Batch Size: 32
\item Dropout: 0.3, 0.2 (two layers)
\item Validation F1-Macro: 0.801
\end{itemize}

\textbf{HMTL Optimal Configuration:}
\begin{itemize}
\item Learning Rate: 0.001
\item Batch Size: 32
\item Task Weighting: $\lambda_1=0.6$ (performance), $\lambda_2=0.4$ (dropout)
\item Validation F1-Macro: 0.729
\end{itemize}

\begin{figure}[h]
\centering
\includegraphics[width=0.95\textwidth]{figures/08_accuracy_heatmap.png}
\caption{\textbf{PPN Hyperparameter Tuning Heatmap.} Accuracy variation across learning rates and batch sizes. Optimal configuration (LR=0.001, BS=32) achieves 77.8\% validation accuracy. Heatmap reveals learning rate 0.001 robustly outperforms alternatives across different batch sizes.}
\label{fig:ppn_heatmap}
\end{figure}

\begin{figure}[h]
\centering
\includegraphics[width=0.95\textwidth]{figures/10_nn_accuracy_heatmap.png}
\caption{\textbf{DPN-A Hyperparameter Tuning Heatmap.} Validation accuracy across learning rates and batch sizes. Optimal configuration (LR=0.001, BS=32) achieves 87.05\% accuracy. Demonstrates superior performance and robustness of attention-based architecture.}
\label{fig:dpna_heatmap}
\end{figure}

\begin{figure}[h]
\centering
\includegraphics[width=0.95\textwidth]{figures/09_ensemble_accuracy_heatmap.png}
\caption{\textbf{HMTL Hyperparameter Tuning Heatmap.} Validation accuracy for multi-task learning configuration. Shows task weighting influence on performance. DPN-A outperforms HMTL, indicating single-task specialization is optimal for this dataset.}
\label{fig:hmtl_heatmap}
\end{figure}

\subsection{Training Algorithm}

\begin{enumerate}
\item Initialize model with Xavier/Glorot initialization
\item For each epoch:
\begin{itemize}
\item Shuffle training data
\item Forward pass through batches
\item Compute loss via appropriate loss function
\item Backpropagation and parameter update via Adam optimizer
\item Validate on validation set
\item Apply learning rate scheduling if plateau detected
\item Check early stopping criterion (patience=20 epochs)
\end{itemize}
\item Return best checkpoint from early stopping
\end{enumerate}

\section{Cross-Validation Protocol}

\subsection{10-Fold Stratified Cross-Validation}

Implemented on combined training+validation sets (N=3,761) to ensure:
\begin{itemize}
\item Robust performance estimation across different data splits
\item Stratified folds preserve class distribution
\item 5 repeated trials with different random seeds
\item Final metrics: mean $\pm$ standard deviation across 50 evaluations (10 folds × 5 repetitions)
\end{itemize}

\section{Reproducibility Provisions}

\subsection{Random Seed Fixation}

All stochastic operations use fixed seeds for complete reproducibility:

\begin{verbatim}
import random
import numpy as np
import torch

random.seed(42)
np.random.seed(42)
torch.manual_seed(42)
torch.cuda.manual_seed_all(42)
\end{verbatim}

\subsection{Documentation and Code Availability}

\begin{itemize}
\item Complete hyperparameter specifications documented in Table 3.1
\item Fixed random seeds (seed=42) for all experiments
\item Training/validation/test split specifications with stratification
\item Detailed loss function implementations with class weights
\item Early stopping and learning rate scheduler configuration
\item Full source code available in supplementary materials
\end{itemize}

\subsection{Environment Reproducibility}

Docker containerization with documented requirements.txt ensuring platform-agnostic reproducibility:

\begin{verbatim}
PyTorch==2.8.0
scikit-learn==1.4.0
pandas==2.2.0
numpy==1.26.0
shap==0.44.0
openai==1.12.0
\end{verbatim}

\section{Computational Performance}

\begin{table}[h]
\centering
\caption{Training Time and Resource Usage}
\label{tab:compute_perf}
\begin{tabular}{lccc}
\toprule
\textbf{Model} & \textbf{Training Time} & \textbf{Inference Time} & \textbf{Model Size} \\
\midrule
PPN & 145 sec (32 epochs) & 0.08 sec/batch & 1.8 MB \\
DPN-A & 128 sec (29 epochs) & 0.07 sec/batch & 1.2 MB \\
HMTL & 224 sec (50 epochs) & 0.09 sec/batch & 2.1 MB \\
\bottomrule
\end{tabular}
\end{table}

\textbf{Inference Throughput:}
\begin{itemize}
\item DPN-A: 6,328 predictions/sec (443 samples in 0.07 seconds)
\item Latency: <1 millisecond per prediction (real-time capable)
\item Batch processing: 100K+ students in <3 minutes
\end{itemize}

\textbf{LLM Integration:}
\begin{itemize}
\item GPT-4 API Latency: 1.8 seconds per recommendation
\item Cost: \$0.03 per student (GPT-4 pricing)
\item Fallback system: Instant rule-based recommendations (zero cost)
\end{itemize}

\section{Summary}

This chapter detailed the technical implementation including Python 3.10+ with PyTorch 2.8.0, comprehensive hyperparameter tuning across 1,728 configurations, optimal learning rates of 0.001 and batch size 32 across all models, systematic 10-fold stratified cross-validation for robust evaluation, and rigorous reproducibility provisions (fixed seeds, documented hyperparameters, Docker containerization). Training completed in 224 seconds maximum per model with inference latency <1 millisecond, demonstrating practical deployability. The next chapter presents experimental results from rigorous evaluation of all proposed architectures.